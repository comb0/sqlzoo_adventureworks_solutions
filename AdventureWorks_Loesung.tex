\documentclass[a4paper]{article}
\usepackage[T1]{fontenc}
\usepackage[utf8]{inputenc}
\usepackage[ngerman]{babel}

\usepackage{listings}
\usepackage{listings}
\usepackage{color}

\definecolor{mygreen}{rgb}{0,0.6,0}
\definecolor{mygray}{rgb}{0.5,0.5,0.5}
\definecolor{mymauve}{rgb}{0.58,0,0.82}

\lstset{ %
  backgroundcolor=\color{white},	   % choose the background color; you must add \usepackage{color} or \usepackage{xcolor}
  basicstyle=\ttfamily,				   % the size of the fonts that are used for the code
  breakatwhitespace=true,			   % sets if automatic breaks should only happen at whitespace
  breaklines=true,					   % sets automatic line breaking
  captionpos=t,						   % sets the caption-position to bottom
  commentstyle=\color{mygreen},		   % comment style
  deletekeywords={...},				   % if you want to delete keywords from the given language
  escapeinside={\%*}{*)},			   % if you want to add LaTeX within your code
  extendedchars=true,				   % lets you use non-ASCII characters; for 8-bits encodings only, does not work with UTF-8
  %frame=single,						   % adds a frame around the code
  keepspaces=true,					   % keeps spaces in text, useful for keeping indentation of code (possibly needs columns=flexible)
  keywordstyle=\color{blue},		   % keyword style
  language=SQL,					   	% the language of the code
  %morekeywords={*,...},				   % if you want to add more keywords to the set
  numbers=left,						   % where to put the line-numbers; possible values are (none, left, right)
  numbersep=5pt,					   % how far the line-numbers are from the code
  numberstyle=\tiny\color{mygray},	   % the style that is used for the line-numbers
  rulecolor=\color{black},			   % if not set, the frame-color may be changed on line-breaks within not-black text (e.g. comments (green here))
  showspaces=false,					   % show spaces everywhere adding particular underscores; it overrides 'showstringspaces'
  showstringspaces=false,			   % underline spaces within strings only
  showtabs=false,					   % show tabs within strings adding particular underscores
  stepnumber=2,						   % the step between two line-numbers. If it's 1, each line will be numbered
  stringstyle=\color{mymauve},		   % string literal style
  tabsize=2,						   % sets default tabsize to 2 spaces
  title=\lstname					   % show the filename of files included with \lstinputlisting; also try caption instead of title
}


\begin{document}
	\title{sqlzoo AdventureWorks Lösung}
	\date{2014-05-22}
	\author{Peter Brantsch}
	\maketitle

	\section{Easy questions}
	\begin{enumerate}
		\item Show the first name and the email address of customer with
		CompanyName 'Bike World'.
			\lstinputlisting{code/easy/1.sql}
		\item Show the CompanyName for all customers with an address in City
		'Dallas'.
			\lstinputlisting{code/easy/2.sql}
		\item How many items with ListPrice more than \$1000 have been sold?
			\lstinputlisting{code/easy/3.sql}
		\item Give the CompanyName of those customers with orders over
		\$100000. Include the subtotal plus tax plus freight.
			\lstinputlisting{code/easy/4.sql}
		\item Find the number of left racing socks ('Racing Socks, L') ordered
		by CompanyName 'Riding Cycles'.
			\lstinputlisting{code/easy/5.sql}
	\end{enumerate}

	\section{Medium questions}
	\begin{enumerate}
		\addtocounter{enumi}{5}
		\item A “Single Item Order” is a customer order where only one item is
		ordered. Show the SalesOrderID and the UnitPrice for every Single Item
		Order.
			\lstinputlisting{code/medium/6.sql}
		\item Where did the racing socks go? List the product name and the
		CompanyName for all Customers who ordered ProductModel 'Racing Socks'.
			\lstinputlisting{code/medium/7.sql}
		\item Show the product description for culture 'fr' for product with
		ProductID 736. 
			\lstinputlisting{code/medium/8.sql}
		\item Use the SubTotal value in SaleOrderHeader to list orders from the
		largest to the smallest. For each order show the CompanyName and the
		SubTotal and the total weight of the order.
			\lstinputlisting{code/medium/9.sql}
		\item How many products in ProductCategory 'Cranksets' have been sold
		to an address in 'London'?
			\lstinputlisting{code/medium/10.sql}
	\end{enumerate}

	\section{Hard questions}
	\begin{enumerate}
		\addtocounter{enumi}{10}
		\item For every customer with a 'Main Office' in Dallas show
		AddressLine1 of the 'Main Office' and AddressLine1 of the 'Shipping'
		address - if there is no shipping address leave it blank. Use one row
		per customer.
			\lstinputlisting{code/hard/11.sql}
		\item For each order show the SalesOrderID and SubTotal calculated
		three ways: \begin{enumerate}
			\item From the SalesOrderHeader 
			\item Sum of OrderQty$\cdot$UnitPrice
			\item Sum of OrderQty$\cdot$ListPrice 
			\end{enumerate}
			\lstinputlisting{code/hard/12.sql}
		\item Show the best selling item by value.
			\lstinputlisting{code/hard/13.sql}
		\item Show how many orders are in the following ranges (in \$):

			\begin{tabular}{r@{ - }rrr}
				\multicolumn{2}{r}{Range} & Num Orders & Total Value \\
				0& 99 \\
				100& 999\\
				1000& 9999\\
				10000&
			\end{tabular}
			\lstinputlisting{code/hard/14.sql}
		\item Identify the three most important cities. Show the break down of
		top level product category against city.
			\lstinputlisting{code/hard/15.sql}
	\end{enumerate}

\end{document}
